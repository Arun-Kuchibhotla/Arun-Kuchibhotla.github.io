%%%%%%%%%%%%%%%%%%%%%%%%%%%%%%%%%%%%%%%%%%%%%%%%%%%%%%%%%%%%%%%%%%%%%%%%
%%%%%%%%%%%%%%%%%%%%%% Simple LaTeX CV Template %%%%%%%%%%%%%%%%%%%%%%%%
%%%%%%%%%%%%%%%%%%%%%%%%%%%%%%%%%%%%%%%%%%%%%%%%%%%%%%%%%%%%%%%%%%%%%%%%

%%%%%%%%%%%%%%%%%%%%%%%%%%%%%%%%%%%%%%%%%%%%%%%%%%%%%%%%%%%%%%%%%%%%%%%%
%% NOTE: If you find that it says                                     %%
%%                                                                    %%
%%                           1 of ??                                  %%
%%                                                                    %%
%% at the bottom of your first page, this means that the AUX file     %%
%% was not available when you ran LaTeX on this source. Simply RERUN  %%
%% LaTeX to get the ``??'' replaced with the number of the last page  %%
%% of the document. The AUX file will be generated on the first run   %%
%% of LaTeX and used on the second run to fill in all of the          %%
%% references.                                                        %%
%%%%%%%%%%%%%%%%%%%%%%%%%%%%%%%%%%%%%%%%%%%%%%%%%%%%%%%%%%%%%%%%%%%%%%%%

%%%%%%%%%%%%%%%%%%%%%%%%%%%% Document Setup %%%%%%%%%%%%%%%%%%%%%%%%%%%%

% Don't like 10pt? Try 11pt or 12pt
\documentclass[10pt]{article}

% The automated optical recognition software used to digitize resume
% information works best with fonts that do not have serifs. This
% command uses a sans serif font throughout. Uncomment both lines (or at
% least the second) to restore a Roman font (i.e., a font with serifs).
%\usepackage{times}
%\renewcommand{\familydefault}{\sfdefault}

% This is a helpful package that puts math inside length specifications
\usepackage{calc}
\usepackage{comment}

% Simpler bibsection for CV sections
% (thanks to natbib for inspiration)
\makeatletter
\newlength{\bibhang}
\setlength{\bibhang}{1em} %1em}
\newlength{\bibsep}
 {\@listi \global\bibsep\itemsep \global\advance\bibsep by\parsep}
\newenvironment{bibsection}%
        {\begin{enumerate}{}{%
%        {\begin{list}{}{%
       \setlength{\leftmargin}{\bibhang}%
       \setlength{\itemindent}{-\leftmargin}%
       \setlength{\itemsep}{\bibsep}%
       \setlength{\parsep}{\z@}%
        \setlength{\partopsep}{0pt}%
        \setlength{\topsep}{0pt}}}
        {\end{enumerate}\vspace{-.6\baselineskip}}
%        {\end{list}\vspace{-.6\baselineskip}}
\makeatother

% Layout: Puts the section titles on left side of page
\reversemarginpar

%
%         PAPER SIZE, PAGE NUMBER, AND DOCUMENT LAYOUT NOTES:
%
% The next \usepackage line changes the layout for CV style section
% headings as marginal notes. It also sets up the paper size as either
% letter or A4. By default, letter was used. If A4 paper is desired,
% comment out the letterpaper lines and uncomment the a4paper lines.
%
% As you can see, the margin widths and section title widths can be
% easily adjusted.
%
% ALSO: Notice that the includefoot option can be commented OUT in order
% to put the PAGE NUMBER *IN* the bottom margin. This will make the
% effective text area larger.
%
% IF YOU WISH TO REMOVE THE ``of LASTPAGE'' next to each page number,
% see the note about the +LP and -LP lines below. Comment out the +LP
% and uncomment the -LP.
%
% IF YOU WISH TO REMOVE PAGE NUMBERS, be sure that the includefoot line
% is uncommented and ALSO uncomment the \pagestyle{empty} a few lines
% below.
%

%% Use these lines for letter-sized paper
\usepackage[paper=letterpaper,
            %includefoot, % Uncomment to put page number above margin
            marginparwidth=1.2in,     % Length of section titles
            marginparsep=.05in,       % Space between titles and text
            margin=1in,               % 1 inch margins
            includemp]{geometry}

%% Use these lines for A4-sized paper
%\usepackage[paper=a4paper,
%            %includefoot, % Uncomment to put page number above margin
%            marginparwidth=30.5mm,    % Length of section titles
%            marginparsep=1.5mm,       % Space between titles and text
%            margin=25mm,              % 25mm margins
%            includemp]{geometry}

%% More layout: Get rid of indenting throughout entire document
\setlength{\parindent}{0in}

\usepackage[shortlabels]{enumitem}

%% Reference the last page in the page number
%
% NOTE: comment the +LP line and uncomment the -LP line to have page
%       numbers without the ``of ##'' last page reference)
%
% NOTE: uncomment the \pagestyle{empty} line to get rid of all page
%       numbers (make sure includefoot is commented out above)
%
\usepackage{fancyhdr,lastpage}
\pagestyle{fancy}
%\pagestyle{empty}      % Uncomment this to get rid of page numbers
\fancyhf{}\renewcommand{\headrulewidth}{0pt}
\fancyfootoffset{\marginparsep+\marginparwidth}
\newlength{\footpageshift}
\setlength{\footpageshift}
          {0.5\textwidth+0.5\marginparsep+0.5\marginparwidth-2in}
\lfoot{\hspace{\footpageshift}%
       \parbox{4in}{\, \hfill %
                    \arabic{page} of \protect\pageref*{LastPage} % +LP
%                    \arabic{page}                               % -LP
                    \hfill \,}}

% Finally, give us PDF bookmarks
\usepackage{color,hyperref}
\definecolor{darkblue}{rgb}{0.0,0.0,0.3}
\hypersetup{colorlinks,breaklinks,
            linkcolor=darkblue,urlcolor=darkblue,
            anchorcolor=darkblue,citecolor=darkblue}

%%%%%%%%%%%%%%%%%%%%%%%% End Document Setup %%%%%%%%%%%%%%%%%%%%%%%%%%%%


%%%%%%%%%%%%%%%%%%%%%%%%%%% Helper Commands %%%%%%%%%%%%%%%%%%%%%%%%%%%%

% The title (name) with a horizontal rule under it
% (optional argument typesets an object right-justified across from name
%  as well)
%
% Usage: \makeheading{name}
%        OR
%        \makeheading[right_object]{name}
%
% Place at top of document. It should be the first thing.
% If ``right_object'' is provided in the square-braced optional
% argument, it will be right justified on the same line as ``name'' at
% the top of the CV. For example:
%
%       \makeheading[\emph{Curriculum vitae}]{Your Name}
%
% will put an emphasized ``Curriculum vitae'' at the top of the document
% as a title. Likewise, a picture could be included:
%
%   \makeheading[\includegraphics[height=1.5in]{my_picutre}]{Your Name}
%
% the picture will be flush right across from the name.
\newcommand{\makeheading}[2][]%
        {\hspace*{-\marginparsep minus \marginparwidth}%
         \begin{minipage}[t]{\textwidth+\marginparwidth+\marginparsep}%
             {\large \bfseries #2 \hfill #1}\\[-0.15\baselineskip]%
                 \rule{\columnwidth}{1pt}%
         \end{minipage}}

% The section headings
%
% Usage: \section{section name}
\renewcommand{\section}[1]{\pagebreak[3]%
    \hyphenpenalty=10000%
    \vspace{1.3\baselineskip}%
    \phantomsection\addcontentsline{toc}{section}{#1}%
    \noindent\llap{\scshape\smash{\parbox[t]{\marginparwidth}{\raggedright #1}}}%
    \vspace{-\baselineskip}\par}

% An itemize-style list with lots of space between items
\newenvironment{outerlist}[1][\enskip\textbullet]%
        {\begin{itemize}[#1,leftmargin=*]}{\end{itemize}%
         \vspace{-.6\baselineskip}}

% An environment IDENTICAL to outerlist that has better pre-list spacing
% when used as the first thing in a \section
\newenvironment{lonelist}[1][\enskip\textbullet]%
        {\begin{list}{#1}{%
        \setlength{\partopsep}{0pt}%
        \setlength{\topsep}{0pt}}}
        {\end{list}\vspace{-.6\baselineskip}}

% An itemize-style list with little space between items
\newenvironment{innerlist}[1][\enskip\textbullet]%
        {\begin{itemize}[#1,leftmargin=*,parsep=0pt,itemsep=0pt,topsep=0pt,partopsep=0pt]}
        {\end{itemize}}

% An environment IDENTICAL to innerlist that has better pre-list spacing
% when used as the first thing in a \section
\newenvironment{loneinnerlist}[1][\enskip\textbullet]%
        {\begin{itemize}[#1,leftmargin=*,parsep=0pt,itemsep=0pt,topsep=0pt,partopsep=0pt]}
        {\end{itemize}\vspace{-.6\baselineskip}}

% To add some paragraph space between lines.
% This also tells LaTeX to preferably break a page on one of these gaps
% if there is a needed pagebreak nearby.
\newcommand{\blankline}{\quad\pagebreak[3]}
\newcommand{\halfblankline}{\quad\vspace{-0.5\baselineskip}\pagebreak[3]}

% Uses hyperref to link DOI
\newcommand\doilink[1]{\href{http://dx.doi.org/#1}{#1}}
\newcommand\doi[1]{doi:\doilink{#1}}

% For \url{SOME_URL}, links SOME_URL to the url SOME_URL
\providecommand*\url[1]{\href{#1}{#1}}
% Same as above, but pretty-prints SOME_URL in teletype fixed-width font
\renewcommand*\url[1]{\href{#1}{\texttt{#1}}}

% For \email{ADDRESS}, links ADDRESS to the url mailto:ADDRESS
\providecommand*\email[1]{\href{mailto:#1}{#1}}
% Same as above, but pretty-prints ADDRESS in teletype fixed-width font
%\renewcommand*\email[1]{\href{mailto:#1}{\texttt{#1}}}

%\providecommand\BibTeX{{\rm B\kern-.05em{\sc i\kern-.025em b}\kern-.08em
%    T\kern-.1667em\lower.7ex\hbox{E}\kern-.125emX}}
%\providecommand\BibTeX{{\rm B\kern-.05em{\sc i\kern-.025em b}\kern-.08em
%    \TeX}}
\providecommand\BibTeX{{B\kern-.05em{\sc i\kern-.025em b}\kern-.08em
    \TeX}}
\providecommand\Matlab{\textsc{Matlab}}

%%%%%%%%%%%%%%%%%%%%%%%% End Helper Commands %%%%%%%%%%%%%%%%%%%%%%%%%%%

%%%%%%%%%%%%%%%%%%%%%%%%% Begin CV Document %%%%%%%%%%%%%%%%%%%%%%%%%%%%

\begin{document}
\makeheading{Arun Kumar Kuchibhotla}

\section{Contact Information}

% NOTE: Mind where the & separators and \\ breaks are in the following
%       table.
%
% ALSO: \rcollength is the width of the right column of the table
%       (adjust it to your liking; default is 1.85in).
%
\newlength{\rcollength}\setlength{\rcollength}{1.6in}%
%
\begin{tabular}[t]{@{}p{\textwidth-\rcollength}p{\rcollength}}
%\href{http://www.cse.osu.edu/}%
%     {Department of Computer Science and Engineering} & \\
%\href{http://www.osu.edu/}{The Ohio State University}
4601 Chester Avenue,    & +1 267 693 3354 \\
Philadelphia, PA -- 19143     & \email{arunku@wharton.upenn.edu}
\end{tabular}

%\section{Objective}

%Insert text here if you want to
%\begin{innerlist}
%\item More information and auxiliary documents can be found at\\\url{http://www.tedpavlic.com/facjobsearch/}
%\end{innerlist}

\section{Research Interests}

Post-selection Inference, Large Sample Theory, Robust Statistics, Non/semi-parametric Statistics, Concentration Inequalities.

\section{Education}
\begin{outerlist}
\item[] Fifth Year Graduate Student, (Advisors: Lawrence D. Brown, Andreas Buja)\\ \href{https://statistics.wharton.upenn.edu/}{\textbf{The Wharton School}},\\ University of Pennsylvania, Philadelphia, USA.\\

\href{www.isical.ac.in}{\textbf{Indian Statistical Institute}},
Kolkata.
\item[] M. Stat. (Hons),
        \begin{innerlist}
        \item Specialization: \emph{Mathematical Statistics and Probability}.
        \item Completed in: 2015.
        \item Class: First class with Distinction. Scored 88\%.
        \end{innerlist}

\item[] B. Stat. (Hons),
        \begin{innerlist}
        \item Major Subject: \emph{Statistics}
        \item Completed in: 2013.
        \item Class: First class with Distinction. Scored 80.5\%.
        \end{innerlist}
\end{outerlist}
%\vspace{.1in}
%\href{http://www.mnsu.edu}{\textbf{Minnesota State University}},
%Mankato, MN
%\begin{outerlist}
%\item[] B.S.,
%        \href{http://www.cset.mnsu.edu/mathstat/}
%             {Mathematics and Statistics} (Double Major), May 2008
%        \begin{innerlist}
%        \item \emph{Summa Cum Laude}
%        \end{innerlist}
%
%\end{outerlist}

% \section{Research Projects Done}
% \begin{innerlist}
% %\item \textbf{Gauss Composition and Bhargava Cubes}\\
% %Dr. \textbf{Vijay. M. Patankar} (ISI, Chennai)\\
% %In this project, I was first introduced to binary quadratic forms, Gauss composition related to them
% %and then the modern reformulation of Gauss composition given by Bhargava in his thesis. Under
% %the guidance of Dr. Patankar, I formulated an algorithm to find Bhargava cubes given two quadratic
% %forms. This project was presented in ISI, Kolkata. This project was accepted for oral presentation in JMM 2013.

% \item\textbf{Estimating Optimal Transformations in Regression}\\
% This project was done as a part of my sixth semester curriculum. In this project, I studied alternating conditional expectation and projection pursuit regression. I developed a test based on bootstrap for testing ACE model and multiple linear regression model. Monte Carlo studies were conducted. Also, a new model incorpora-ting both ACE and PPR was proposed and this new model was shown to be better for prediction also in Monte Carlo studies. This project was presented in ISI, Chennai.    
% %\item\textbf{New Family of Divergences for Robust Statistical Inference} \\
% %Dr. \textbf{Ayanendranth Basu} (ISRU, ISI, Kolkata)\hfill May-July 2013\\
% %In this project, I was first introduced to statistical inference based on minimum distance estimation. There is a special class of statistical distances called divergences. Under guidance of Dr. Basu, I constructed a new parametric class of divergences which will allow robust estimation and also contains many of the well-known distances as a special cases. This manuscript is under preparation.
% \item\textbf{Robust Statistical Inference using Characteristic Functions}\\
% Dr. \textbf{Ayanendranath Basu} (ISRU, ISI, Kolkata)\\
% This project was done along with my co-student Promit Ghosal (currently a graduate student at Columbia University). Statistical divergences and statistical distances considered in literature are between densities. Under guidance of Dr. Basu, we consider $L_2$ distance between characteristic functions to estimate parameters. It was shown by Monte Carlo studies that efficiency of the estimators using other distances is very poor. Robustness properties of $L_2$ distance based were studied. This project was presented at ICORS 2016.
% \item \textbf{A General Set Up for Minimum Disparity Estimation}\\
% Dr. \textbf{Ayanendranath Basu} (ISRU, ISI, Kolkata)\\
% Minimum disparity estimation was demonstrated in literature as a potential alternati-ve to the maximum likelihood estimation as it achieves full asymptotic efficiency at the model and is stable against deviations from the model. But a general framework for continuous families was missing and was formulated and proved in this project. This manuscript was published in statistics and probability letters (2014). 
% \item \textbf{Non-Parametric Curve Estimation Using Statistical Divergence}\\
% Dr. \textbf{Ayanendranath Basu} (ISRU, ISI, Kolkata)\\
% Statistical divergences were extensively studied in parametric models and were shown to be robust at an expense of a little loss in asymptotic efficiency. The goal of this project was to use statistical divergences for non-parametric curve estimation. Currently, we are considering the estimation of regression function and density function. If we can demonstrate the properties of regression function, then by asymptotic equivalence theory of regression function estimation problem and density function estimation problem, we can demonstrate similar properties for density estimate. The goal of this project is to consider non-parametric estimators obtained as extending the idea of penalized log-likelihood estimation and local likelihood estimation using divergences instead of likelihood. In Monte Carlo studies, we demonstrated that these estimators are robust to outliers in response variable. 
% \item \textbf{Statistical Inference based on Bridge Divergences}\\
% Dr. \textbf{Ayandendranath Basu} (ISRU, ISI, Kolkata)\\
% This project was done in collaboration with Somabha Mukherjee (currently a doctoral student at Wharton). In this paper, we present a generalized family of divergences incorporating the two well-known classes called DPD and LDPD; this family provides a smooth bridge between the DPD and the LDPD measures. This family helps to clarify and settle several longstanding issues in the relation between the important
% families of DPD and LDPD, apart from being important tools in different areas of statistical inference on their own right. The manuscript is ready and will be available soon in arxiv.
% \item \textbf{Efficient Estimation in Single Index Models using Smoothing Splines}\\
% Dr. \textbf{Bodhisattva Sen} (Columbia University)\\
% This project was done along with Rohit Kumar Patra, currently an assistant professor at university of Florida. Single index models with an unknown link function offer a good alternative to non-parametric regression with high dimensional covariate vector. In this project, we study an estimation procedure using smoothing splines. We studied the large sample properties of the function estimate and also the index vector estimate in this cases. Monte Carlo studies were conducted to compare the estimation procedures. This manuscript is submitted to Bernoulli and is available in arxiv.    
% \item \textbf{Efficient Estimation in Single Index Models with Convex Link}\\
% Dr. \textbf{Bodhisattva Sen} (Columbia University)\\
% This project was done along with Rohit Kumar Patra. Single index models with an unknown link function offer a good alternative to non-parametric regression with high dimensional covariate vector. In this project, we study two different estimation procedures with convexity constraint. First being the penalized least squares with an additional constraint for convexity. Second being the convex least squares. We studied the large sample properties of the convex function estimate and also the index vector estimate in both cases. Monte Carlo studies were conducted to compare the estimation procedures. This manuscript is currently in preprint form and will be submitted soon. 
% \item \textbf{Models as approximations: a general theory}\\
% This project was in collaboration with the Wharton linear models group. The paper provides a theory and fundamental understanding of mis-specification in general regression models. Some asymptotic theory and foundational concepts were discussed. Manuscript currently available in arxiv and accepted for publication in Statistical Science.
% \item \textbf{Valid Post-selection Inference in Assumption-lean Linear Regression}\\
% This project was done in collaboration with the Wharton linear models group. The paper indicates how one can perform valid inference after choosing the model based on the same data. This is the first such inference procedure available. The manuscript is available in preprint form and will be made available in arxiv soon.
% %\item \textbf{A Minimum Distance Weighted Likelihood Method of Estimation}\\
% %Dr. \textbf{Ayanendranath Basu} (ISRU, ISI, Kolkata)\hfill July--August 2014\\
% %In literature, minimum disparity estimators and weighted likelihood estimators were studied as robust alternatives to the maximum likelihood estimation. These two methods have distinct identities. Despite their similarities, they have some basic differences. In this project we proposed a method of estimation which is simultaneous-ly a minimum disparity method and a weighted likelihood method, and may be viewed as combining the positive aspects of both. We studied the properties of the corresponding minimum disparity weighted likelihood (MDWL) estimators, and illustrated their properties through real data examples and simulations. This project was presented at the International Conference on Robust Statistics 2014, in Germany. This manuscript is currently under review.
% \end{innerlist}
% \section{Ongoing Projects}
% \begin{outerlist}
% \item \textbf{Integral Approximation and some finite sample results}\\
% Dr. \textbf{Abhishek Chakrabortty} (Wharton)\\
% This project revisits the classical problem of approximating an integral of a function based on evaluations at independent points randomly generated with respect to some distribution. We extend previous results and provide finite sample guarantees.
% \end{outerlist}
\section{Journal Publications}
\vspace{-.1275in}
\begin{bibsection}
    \item \textbf{Kuchibhotla A. K.} and Basu A. (2015) ``A General Set Up for Minimum Disparity Estimation." \emph{Statistics and Probability Letters},  Vol. 96, 68-74, 2015.
    \item \textbf{Kuchibhotla A. K.} and Basu A. (2017) ``On The Asymptotics of Minimum Disparity Estimation.'' \emph{TEST: An Official Journal of the Spanish Society of Statistics and Operations Research, 22.}
    \item \textbf{Kuchibhotla A. K.}, Mukherjee S. and Basu A. (2017) ``Statistical inference based on bridge divergences.'' \emph{Annals of the Institute of Statistical Mathematics, 71 (3), 627-656.}
    \item Berk R., Buja A., Brown L. D., George E. I., \textbf{Kuchibhotla A. K.}, Su W. J., Zhao L. H., (2018) ``Assumption Lean Regression'' \emph{The American Statistician, To appear.}
    \item Bellec P., and \textbf{Kuchibhotla A. K.}, (2019) ``First order expansion of convex regularized estimators'' \emph{Advances in Neural Information Processing Systems 33, To appear.}
\end{bibsection}

\section{Preprints}
\vspace{-.1275in}
\begin{bibsection}
	\item \textbf{Kuchibhotla A. K.} and Parta R. K. (2016) ``Efficient Estimation in Single Index Models through Smoothing splines'' \emph{arXiv preprint arXiv:1612.00068.}
	\item Buja A., Berk R., Brown L. D., George E. I., \textbf{Kuchibhotla A. K.}, Zhao L. H. (2016) ``Models as Approximations --- Part II: A General Theory of Model-Robust Regression'' \emph{arXiv preprint arXiv:1612.03257.}
	\item \textbf{Kuchibhotla A. K.}, Patra R. K., Sen B. (2017) ``Least Squares Estimation in a Single Index Model with Convex Lipschitz link'' \emph{arXiv preprint arXiv:1708.00145.}
  \item \textbf{Kuchibhotla A. K.}, Brown L. D., Buja A., George E.I., Zhao L.H. (2017) ``Valid Post-selection Inference in Assumption-lean Linear Regression." \emph{arXiv preprint arXiv:1806.04119.}
  \item \textbf{Kuchibhotla A. K.}, Brown L. D., Buja A., George E.I., Zhao L.H. (2018) ``A Model Free Perspective for Linear Regression: Uniform-in-model Bounds for Post Selection Inference.'' \emph{arXiv preprint arXiv:1802.05801.}
  \item \textbf{Kuchibhotla A. K.}, Chakrabortty A. (2018) ``Moving Beyond Sub-Gaussianity in High-Dimensional Statistics: Applications in Covariance Estimation and Linear Regression.'' \emph{arXiv preprint arXiv:1804.02605.}
  \item Chakrabortty A. and \textbf{Kuchibhotla A. K.} (2018) ``Tail Bounds for Canonical U-Statistics and U-Processes with Unbounded Kernels.''
  \item \textbf{Kuchibhotla A. K.}, Brown L. D., Buja A. (2018) ``Model-free Study of Ordinary Least Squares Linear Regression.'' \emph{arxiv preprint arxiv:1809.10538.}
  \item Banerjee D., \textbf{Kuchibhotla A. K.}, Muhkerjee S. (2019) ``High-dimensional CLT: Improvements, Non-uniform Extensions and Large Deviations'' \emph{arXiv preprint} \emph{arXiv:1806.06153.}
  \item \textbf{Kuchibhotla A. K.} (2018) ``Deterministic Inequalities for Smooth M-estimators.'' \emph{arxiv preprint arxiv:1809.05172.} 
  \item \textbf{Kuchibhotla A. K.} and Patra R. K. (2019) ``On Least Squares Estimation under Heteroscedastic and Heavy-Tailed Errors.'' \emph{arxiv preprint arXiv:1909.02088.}
\end{bibsection}

%\section{Submitted Journal Publications}
%\vspace{-.125in}
%\begin{bibsection}
%    \item Toomey, T.L., Erickson, D.J., Carlin, B.P., Lenk, K.M., {\bf Quick, H.S.}, and Harwood, E.M. ``Do neighborhood attributes moderate the relationship between alcohol establishment density and crime?" 2012. Submitted to \emph{Prevention Science}.
%\end{bibsection}

% Add a little space to nudge next ``Conference Publications'' marginpar
% down to make room for tall ``Submitted Journal Publications''
% marginpar. If there are enough submitted journal publications, this
% space will not be needed (and should be removed).
%\vspace{0.1in}

%\section{Papers in Preparation}
%\vspace{-.1in}
%\begin{bibsection}
%%    \item Toomey, T.L., Erickson, D.J., Carlin, B.P., Lenk, K.M., {\bf Quick, H.S.}, and Harwood, E.M. ``Do neighborhood attributes moderate the relationship between alcohol establishment density and crime?"
%    \item {\bf Quick, H.}, Banerjee, S., and Carlin, B.P. ``Heteroscedastic variances in areally referenced temporal processes with an application to California asthma hospitalization data.''
%
%    \item {\bf Quick, H.}, Carlin, B.P., and Banerjee, S. ``Space-time Gaussian process modeling of temporal air pollution gradients."
%\end{bibsection}

\section{Presentations}
\begin{innerlist}
\item Invited talk at MCP 2019, National Taiwan University.\\ URL: \url{https://2019mcp.smartevent.com.tw/}
\item Invited talk at ICSA 2019, Nankai University.
\item Shared a talk with Andreas Buja at WHOA-PSI-3, 2018.\\ URL: \url{https://www.math.wustl.edu/~kuffner/WHOA-PSI-3.html} 
\item Invited talk at ``Workshop Model Selection, Regularization, and Inference'' 2018 (Represented Larry Brown).\\ URL: \url{https://www.univie.ac.at/seam/inference2018/}
\item Shared a talk with Andreas Buja at SLDSC 2018.\\ URL: \url{https://publish.illinois.edu/sldsc2018/}
\item Invited talk at WHOA-PSI-2, 2017 (Represented Larry Brown).\\ URL: \url{https://www.math.wustl.edu/~kuffner/WHOA-PSI-2.html}
\item Contributed session presentation at the International Conference on Robust Statistics (ICORS) 2015, Kolkata, India.
\item Contributed session presentation at the International Conference on Robust Statistics (ICORS) 2014, Halle, Germany.
\item Two seminars in Theoretical Statistics and Mathematics Unit, ISI, Kolkata on August 7 and 13, 2012 based on my project under Dr. Patankar.
\item A seminar in ISI, Chennai on July 3, 2013 based on my project Estimating Optimal Transformations for Regression.
\end{innerlist}

\section{Academic Achievements}
\begin{innerlist}
\item Got second prize in \href{http://www.isi-web.org/index.php/activities/awards/isi-awards/tinbergen-award}{Jan Tinbergen} competition for young statisticians from developing countries.\\
URL: \url{https://isi-web.org/index.php/activities/awards/isi-awards/tinbergen-award}
\item Awarded Kishore Vaigyanik Protsahan Yojana scholarship since April 2011. (From Indian Institute of Science)
\item Inspire Scholar since April 2011. (From Department of Science and Technology, India)
% \item Selected in IIT-Joint Entrance Exam 2010 (All India Rank - 2469), AIEEE 2010 (AIR-2604), BITSAT 2010 (Score-343)
% \item Stood district second in mathematics competition conducted by Ramanujan Mathema-tics Academy
% in 2008.     
\end{innerlist}


\section{Programming and Scripting}
\begin{innerlist}
\item C -- Basic Programming,
\item R -- Proficient,
\item Sage -- Basic Programming,
\item \LaTeXe -- Proficient
\end{innerlist}
% I can effectively work with both windows and Linux operating systems. 
%\section{Service}
%Recruiting Committee, Division of Biostatistics \hfill {May 2010 -- Present}
%\begin{innerlist}
%    \item Assist with planning of annual Division of Biostatistics Open House and Admitted Student Visit Days
%    \item Meet with prospective and admitted students %; answer questions from a student's perspective
%\end{innerlist}
%
%\halfblankline
%
%Student Member of Search Committee for the \hfill {June 2010 -- Aug 2010}\\
%SPH Coordinator of Recruitment and Student Leadership
%\begin{innerlist}
%    \item Assisted in job search for the SPH Coordinator of Recruitment and Student Leadership
%    \item Reviewed applications, conducted interviews
%\end{innerlist}
%
%\section{References}
%
%Bradley P.\ Carlin
%\begin{innerlist}
%\item[] Mayo Professor in Public Health, Division Head \hfill {Phone: 612-624-6646}\\
%Division of Biostatistics \hfill{E-mail: carli002@umn.edu}\\
%University of Minnesota
%\end{innerlist}
%
%\halfblankline
%
%Sudipto Banerjee
%\begin{innerlist}
%\item[] Professor \hfill {Phone: 612-624-0624}\\
%Division of Biostatistics \hfill{E-mail: baner009@umn.edu}\\
%University of Minnesota
%\end{innerlist}
%
%\halfblankline
%
%Traci Toomey
%\begin{innerlist}
%\item[] Professor \hfill {Phone: 612-626-9070}\\
%Division of Epidemiology \hfill{E-mail: toome001@umn.edu}\\
%University of Minnesota
%\end{innerlist}
%
%%\section{Hardware and Software Skills}
%\begin{comment}
%Computer Programming:
%%
%\begin{innerlist}
%    \item C, C$+$$+$, Java, JavaScript, NetLogo, Pascal, Perl, PHP,
%        Lisp, UNIX shell scripting (including POSIX.2), GNU make,
%        AppleScript, SQL, MySQL, \Matlab, Maple, Mathematica, and others
%\end{innerlist}
%
%\halfblankline
%\end{comment}

\end{document}

%%%%%%%%%%%%%%%%%%%%%%%%%% End CV Document %%%%%%%%%%%%%%%%%%%%%%%%%%%%%

%----------------------------------------------------------------------%
% The following is copyright and licensing information for
% redistribution of this LaTeX source code; it also includes a liability
% statement. If this source code is not being redistributed to others,
% it may be omitted. It has no effect on the function of the above code.
%----------------------------------------------------------------------%
% Copyright (c) 2007, 2008, 2009, 2010, 2011 by Theodore P. Pavlic
%
% Unless otherwise expressly stated, this work is licensed under the
% Creative Commons Attribution-Noncommercial 3.0 United States License. To
% view a copy of this license, visit
% http://creativecommons.org/licenses/by-nc/3.0/us/ or send a letter to
% Creative Commons, 171 Second Street, Suite 300, San Francisco,
% California, 94105, USA.
%
% THE SOFTWARE IS PROVIDED "AS IS", WITHOUT WARRANTY OF ANY KIND, EXPRESS
% OR IMPLIED, INCLUDING BUT NOT LIMITED TO THE WARRANTIES OF
% MERCHANTABILITY, FITNESS FOR A PARTICULAR PURPOSE AND NONINFRINGEMENT.
% IN NO EVENT SHALL THE AUTHORS OR COPYRIGHT HOLDERS BE LIABLE FOR ANY
% CLAIM, DAMAGES OR OTHER LIABILITY, WHETHER IN AN ACTION OF CONTRACT,
% TORT OR OTHERWISE, ARISING FROM, OUT OF OR IN CONNECTION WITH THE
% SOFTWARE OR THE USE OR OTHER DEALINGS IN THE SOFTWARE.
%----------------------------------------------------------------------%
